%% bare_jrnl.tex
%% V1.4b
%% 2015/08/26
%% by Michael Shell
%% see http://www.michaelshell.org/
%% for current contact information.
%%
%% This is a skeleton file demonstrating the use of IEEEtran.cls
%% (requires IEEEtran.cls version 1.8b or later) with an IEEE
%% journal paper.
%%
%% Support sites:
%% http://www.michaelshell.org/tex/ieeetran/
%% http://www.ctan.org/pkg/ieeetran
%% and
%% http://www.ieee.org/

%%*************************************************************************
%% Legal Notice:
%% This code is offered as-is without any warranty either expressed or
%% implied; without even the implied warranty of MERCHANTABILITY or
%% FITNESS FOR A PARTICULAR PURPOSE! 
%% User assumes all risk.
%% In no event shall the IEEE or any contributor to this code be liable for
%% any damages or losses, including, but not limited to, incidental,
%% consequential, or any other damages, resulting from the use or misuse
%% of any information contained here.
%%
%% All comments are the opinions of their respective authors and are not
%% necessarily endorsed by the IEEE.
%%
%% This work is distributed under the LaTeX Project Public License (LPPL)
%% ( http://www.latex-project.org/ ) version 1.3, and may be freely used,
%% distributed and modified. A copy of the LPPL, version 1.3, is included
%% in the base LaTeX documentation of all distributions of LaTeX released
%% 2003/12/01 or later.
%% Retain all contribution notices and credits.
%% ** Modified files should be clearly indicated as such, including  **
%% ** renaming them and changing author support contact information. **
%%*************************************************************************


% *** Authors should verify (and, if needed, correct) their LaTeX system  ***
% *** with the testflow diagnostic prior to trusting their LaTeX platform ***
% *** with production work. The IEEE's font choices and paper sizes can   ***
% *** trigger bugs that do not appear when using other class files.       ***                          ***
% The testflow support page is at:
% http://www.michaelshell.org/tex/testflow/



\documentclass[journal]{IEEEtran}
%
% If IEEEtran.cls has not been installed into the LaTeX system files,
% manually specify the path to it like:
% \documentclass[journal]{../sty/IEEEtran}

%%%% packages and definitions (optional)
\usepackage{graphicx} % allows inclusion of graphics
\usepackage{booktabs} % nice rules (thick lines) for tables
\usepackage{microtype} % improves typography for PDF
\usepackage{xcolor}
\usepackage{amsmath}
\usepackage{tabulary}
\usepackage{amssymb}
\usepackage{hyperref}

% \usepackage{caption}
%\usepackage{subcaption}
\usepackage{bm}
\usepackage{float}
\usepackage{tikz}
\usepackage{verbatim}
\usetikzlibrary{arrows,shapes}
 
\usepackage{braket}
\usepackage[figuresright]{rotating}

%% ----Putting 2 pic into 1
% \documentclass{articles}
\usepackage{graphicx}
\newcommand{\SN}{S$_N$}
\renewcommand{\vec}[1]{\bm{#1}} %vector is bold italic
\newcommand{\vd}{\bm{\cdot}} % slightly bold vector dot
\newcommand{\grad}{\vec{\nabla}} % gradient
\newcommand{\ud}{\mathop{}\!\mathrm{d}} % upright derivative symbol
\providecommand{\e}[1]{\ensuremath{\vd 10^{#1}}}
\newcommand{\oper}[1]{\mathcal{#1}}
\newcommand{\EQ}[1]{Eq.~(\ref{#1})}               %-- Eq. (refeq)
\newcommand{\EQUATION}[1]{Equation~(\ref{#1})}    %-- Equation (refeq)
\newcommand{\FIG}[1]{Fig.~\ref{#1}}               %-- Fig. refig
\newcommand{\FIGURE}[1]{Figure~\ref{#1}}          %-- Figure refig
\newcommand{\TAB}[1]{Table~\ref{#1}}              %-- Table tablref
\newcommand{\EQS}[2]{Eqs.~(\ref{#1})--(\ref{#2})}            %-- Eqs. (refeqs)
\newcommand{\EQUATIONS}[2]{Equations~(\ref{#1})--(\ref{#2})} %-- Eqs. (refeqs)
\newcommand{\EQSTWO}[2]{Eqs.~(\ref{#1})~and~(\ref{#2})} %-- Eqs. (refeqs)
\newcommand{\EQUATIONSTWO}[2]{Equations~(\ref{#1})~and~(\ref{#2})}             
%-- Eqs. (refeqs
\newcommand{\BOXEQ}[1]{\mbox{\fboxsep=.13in $$
    \framebox{#1} $$ } }    %-- box around equation
\newcommand{\SEC}[1]{Section~\ref{#1}}               %-- Eq. (refeq)
\newcommand{\REF}[1]{Ref.~\citen{#1}}               %-- Eq. (refeq)
\DeclareMathOperator*{\dotp}{{\scriptscriptstyle \stackrel{\bullet}{{}}}}



% Some very useful LaTeX packages include:
% (uncomment the ones you want to load)


% *** MISC UTILITY PACKAGES ***
%
%\usepackage{ifpdf}
% Heiko Oberdiek's ifpdf.sty is very useful if you need conditional
% compilation based on whether the output is pdf or dvi.
% usage:
% \ifpdf
%   % pdf code
% \else
%   % dvi code
% \fi
% The latest version of ifpdf.sty can be obtained from:
% http://www.ctan.org/pkg/ifpdf
% Also, note that IEEEtran.cls V1.7 and later provides a builtin
% \ifCLASSINFOpdf conditional that works the same way.
% When switching from latex to pdflatex and vice-versa, the compiler may
% have to be run twice to clear warning/error messages.






% *** CITATION PACKAGES ***
%
%\usepackage{cite}
% cite.sty was written by Donald Arseneau
% V1.6 and later of IEEEtran pre-defines the format of the cite.sty package
% \cite{} output to follow that of the IEEE. Loading the cite package will
% result in citation numbers being automatically sorted and properly
% "compressed/ranged". e.g., [1], [9], [2], [7], [5], [6] without using
% cite.sty will become [1], [2], [5]--[7], [9] using cite.sty. cite.sty's
% \cite will automatically add leading space, if needed. Use cite.sty's
% noadjust option (cite.sty V3.8 and later) if you want to turn this off
% such as if a citation ever needs to be enclosed in parenthesis.
% cite.sty is already installed on most LaTeX systems. Be sure and use
% version 5.0 (2009-03-20) and later if using hyperref.sty.
% The latest version can be obtained at:
% http://www.ctan.org/pkg/cite
% The documentation is contained in the cite.sty file itself.






% *** GRAPHICS RELATED PACKAGES ***
%
\ifCLASSINFOpdf
  % \usepackage[pdftex]{graphicx}
  % declare the path(s) where your graphic files are
  % \graphicspath{{../pdf/}{../jpeg/}}
  % and their extensions so you won't have to specify these with
  % every instance of \includegraphics
  % \DeclareGraphicsExtensions{.pdf,.jpeg,.png}
\else
  % or other class option (dvipsone, dvipdf, if not using dvips). graphicx
  % will default to the driver specified in the system graphics.cfg if no
  % driver is specified.
  % \usepackage[dvips]{graphicx}
  % declare the path(s) where your graphic files are
  % \graphicspath{{../eps/}}
  % and their extensions so you won't have to specify these with
  % every instance of \includegraphics
  % \DeclareGraphicsExtensions{.eps}
\fi
% graphicx was written by David Carlisle and Sebastian Rahtz. It is
% required if you want graphics, photos, etc. graphicx.sty is already
% installed on most LaTeX systems. The latest version and documentation
% can be obtained at: 
% http://www.ctan.org/pkg/graphicx
% Another good source of documentation is "Using Imported Graphics in
% LaTeX2e" by Keith Reckdahl which can be found at:
% http://www.ctan.org/pkg/epslatex
%
% latex, and pdflatex in dvi mode, support graphics in encapsulated
% postscript (.eps) format. pdflatex in pdf mode supports graphics
% in .pdf, .jpeg, .png and .mps (metapost) formats. Users should ensure
% that all non-photo figures use a vector format (.eps, .pdf, .mps) and
% not a bitmapped formats (.jpeg, .png). The IEEE frowns on bitmapped formats
% which can result in "jaggedy"/blurry rendering of lines and letters as
% well as large increases in file sizes.
%
% You can find documentation about the pdfTeX application at:
% http://www.tug.org/applications/pdftex





% *** MATH PACKAGES ***
%
%\usepackage{amsmath}
% A popular package from the American Mathematical Society that provides
% many useful and powerful commands for dealing with mathematics.
%
% Note that the amsmath package sets \interdisplaylinepenalty to 10000
% thus preventing page breaks from occurring within multiline equations. Use:
%\interdisplaylinepenalty=2500
% after loading amsmath to restore such page breaks as IEEEtran.cls normally
% does. amsmath.sty is already installed on most LaTeX systems. The latest
% version and documentation can be obtained at:
% http://www.ctan.org/pkg/amsmath





% *** SPECIALIZED LIST PACKAGES ***
%
%\usepackage{algorithmic}
% algorithmic.sty was written by Peter Williams and Rogerio Brito.
% This package provides an algorithmic environment fo describing algorithms.
% You can use the algorithmic environment in-text or within a figure
% environment to provide for a floating algorithm. Do NOT use the algorithm
% floating environment provided by algorithm.sty (by the same authors) or
% algorithm2e.sty (by Christophe Fiorio) as the IEEE does not use dedicated
% algorithm float types and packages that provide these will not provide
% correct IEEE style captions. The latest version and documentation of
% algorithmic.sty can be obtained at:
% http://www.ctan.org/pkg/algorithms
% Also of interest may be the (relatively newer and more customizable)
% algorithmicx.sty package by Szasz Janos:
% http://www.ctan.org/pkg/algorithmicx




% *** ALIGNMENT PACKAGES ***
%
%\usepackage{array}
% Frank Mittelbach's and David Carlisle's array.sty patches and improves
% the standard LaTeX2e array and tabular environments to provide better
% appearance and additional user controls. As the default LaTeX2e table
% generation code is lacking to the point of almost being broken with
% respect to the quality of the end results, all users are strongly
% advised to use an enhanced (at the very least that provided by array.sty)
% set of table tools. array.sty is already installed on most systems. The
% latest version and documentation can be obtained at:
% http://www.ctan.org/pkg/array


% IEEEtran contains the IEEEeqnarray family of commands that can be used to
% generate multiline equations as well as matrices, tables, etc., of high
% quality.




% *** SUBFIGURE PACKAGES ***
%\ifCLASSOPTIONcompsoc
%  \usepackage[caption=false,font=normalsize,labelfont=sf,textfont=sf]{subfig}
%\else
%  \usepackage[caption=false,font=footnotesize]{subfig}
%\fi
% subfig.sty, written by Steven Douglas Cochran, is the modern replacement
% for subfigure.sty, the latter of which is no longer maintained and is
% incompatible with some LaTeX packages including fixltx2e. However,
% subfig.sty requires and automatically loads Axel Sommerfeldt's caption.sty
% which will override IEEEtran.cls' handling of captions and this will result
% in non-IEEE style figure/table captions. To prevent this problem, be sure
% and invoke subfig.sty's "caption=false" package option (available since
% subfig.sty version 1.3, 2005/06/28) as this is will preserve IEEEtran.cls
% handling of captions.
% Note that the Computer Society format requires a larger sans serif font
% than the serif footnote size font used in traditional IEEE formatting
% and thus the need to invoke different subfig.sty package options depending
% on whether compsoc mode has been enabled.
%
% The latest version and documentation of subfig.sty can be obtained at:
% http://www.ctan.org/pkg/subfig




% *** FLOAT PACKAGES ***
%
%\usepackage{fixltx2e}
% fixltx2e, the successor to the earlier fix2col.sty, was written by
% Frank Mittelbach and David Carlisle. This package corrects a few problems
% in the LaTeX2e kernel, the most notable of which is that in current
% LaTeX2e releases, the ordering of single and double column floats is not
% guaranteed to be preserved. Thus, an unpatched LaTeX2e can allow a
% single column figure to be placed prior to an earlier double column
% figure.
% Be aware that LaTeX2e kernels dated 2015 and later have fixltx2e.sty's
% corrections already built into the system in which case a warning will
% be issued if an attempt is made to load fixltx2e.sty as it is no longer
% needed.
% The latest version and documentation can be found at:
% http://www.ctan.org/pkg/fixltx2e


%\usepackage{stfloats}
% stfloats.sty was written by Sigitas Tolusis. This package gives LaTeX2e
% the ability to do double column floats at the bottom of the page as well
% as the top. (e.g., "\begin{figure*}[!b]" is not normally possible in
% LaTeX2e). It also provides a command:
%\fnbelowfloat
% to enable the placement of footnotes below bottom floats (the standard
% LaTeX2e kernel puts them above bottom floats). This is an invasive package
% which rewrites many portions of the LaTeX2e float routines. It may not work
% with other packages that modify the LaTeX2e float routines. The latest
% version and documentation can be obtained at:
% http://www.ctan.org/pkg/stfloats
% Do not use the stfloats baselinefloat ability as the IEEE does not allow
% \baselineskip to stretch. Authors submitting work to the IEEE should note
% that the IEEE rarely uses double column equations and that authors should try
% to avoid such use. Do not be tempted to use the cuted.sty or midfloat.sty
% packages (also by Sigitas Tolusis) as the IEEE does not format its papers in
% such ways.
% Do not attempt to use stfloats with fixltx2e as they are incompatible.
% Instead, use Morten Hogholm'a dblfloatfix which combines the features
% of both fixltx2e and stfloats:
%
% \usepackage{dblfloatfix}
% The latest version can be found at:
% http://www.ctan.org/pkg/dblfloatfix




%\ifCLASSOPTIONcaptionsoff
%  \usepackage[nomarkers]{endfloat}
% \let\MYoriglatexcaption\caption
% \renewcommand{\caption}[2][\relax]{\MYoriglatexcaption[#2]{#2}}
%\fi
% endfloat.sty was written by James Darrell McCauley, Jeff Goldberg and 
% Axel Sommerfeldt. This package may be useful when used in conjunction with 
% IEEEtran.cls'  captionsoff option. Some IEEE journals/societies require that
% submissions have lists of figures/tables at the end of the paper and that
% figures/tables without any captions are placed on a page by themselves at
% the end of the document. If needed, the draftcls IEEEtran class option or
% \CLASSINPUTbaselinestretch interface can be used to increase the line
% spacing as well. Be sure and use the nomarkers option of endfloat to
% prevent endfloat from "marking" where the figures would have been placed
% in the text. The two hack lines of code above are a slight modification of
% that suggested by in the endfloat docs (section 8.4.1) to ensure that
% the full captions always appear in the list of figures/tables - even if
% the user used the short optional argument of \caption[]{}.
% IEEE papers do not typically make use of \caption[]'s optional argument,
% so this should not be an issue. A similar trick can be used to disable
% captions of packages such as subfig.sty that lack options to turn off
% the subcaptions:
% For subfig.sty:
% \let\MYorigsubfloat\subfloat
% \renewcommand{\subfloat}[2][\relax]{\MYorigsubfloat[]{#2}}
% However, the above trick will not work if both optional arguments of
% the \subfloat command are used. Furthermore, there needs to be a
% description of each subfigure *somewhere* and endfloat does not add
% subfigure captions to its list of figures. Thus, the best approach is to
% avoid the use of subfigure captions (many IEEE journals avoid them anyway)
% and instead reference/explain all the subfigures within the main caption.
% The latest version of endfloat.sty and its documentation can obtained at:
% http://www.ctan.org/pkg/endfloat
%
% The IEEEtran \ifCLASSOPTIONcaptionsoff conditional can also be used
% later in the document, say, to conditionally put the References on a 
% page by themselves.




% *** PDF, URL AND HYPERLINK PACKAGES ***
%
%\usepackage{url}
% url.sty was written by Donald Arseneau. It provides better support for
% handling and breaking URLs. url.sty is already installed on most LaTeX
% systems. The latest version and documentation can be obtained at:
% http://www.ctan.org/pkg/url
% Basically, \url{my_url_here}.




% *** Do not adjust lengths that control margins, column widths, etc. ***
% *** Do not use packages that alter fonts (such as pslatex).         ***
% There should be no need to do such things with IEEEtran.cls V1.6 and later.
% (Unless specifically asked to do so by the journal or conference you plan
% to submit to, of course. )


% correct bad hyphenation here
\hyphenation{op-tical net-works semi-conduc-tor}


\begin{document}
%
% paper title
% Titles are generally capitalized except for words such as a, an, and, as,
% at, but, by, for, in, nor, of, on, or, the, to and up, which are usually
% not capitalized unless they are the first or last word of the title.
% Linebreaks \\ can be used within to get better formatting as desired.
% Do not put math or special symbols in the title.
\title{Application of MAXED and Gravel Unfolding Methods to In-Core, Fission Chamber, Neutron Spectrometry}

\author{John Boyington, Jeremy Roberts% <-this % stops a space
\thanks{J. Boyington is with the Department
of Mechanical and Nuclear Engineering, Kansas State University, Manhattan,
KS, 66502 (email: mjcb@ksu.edu).}
% <-this % stops a space
}

% note the % following the last \IEEEmembership and also \thanks - 
% these prevent an unwanted space from occurring between the last author name
% and the end of the author line. i.e., if you had this:
% 
% \author{....lastname \thanks{...} \thanks{...} }
%                     ^------------^------------^----Do not want these spaces!
%
% a space would be appended to the last name and could cause every name on that
% line to be shifted left slightly. This is one of those "LaTeX things". For
% instance, "\textbf{A} \textbf{B}" will typeset as "A B" not "AB". To get
% "AB" then you have to do: "\textbf{A}\textbf{B}"
% \thanks is no different in this regard, so shield the last } of each \thanks
% that ends a line with a % and do not let a space in before the next \thanks.
% Spaces after \IEEEmembership other than the last one are OK (and needed) as
% you are supposed to have spaces between the names. For what it is worth,
% this is a minor point as most people would not even notice if the said evil
% space somehow managed to creep in.



% The paper headers
% \markboth{Journal of \LaTeX\ Class Files,~Vol.~14, No.~8, August~2015}%
% {Shell \MakeLowercase{\textit{et al.}}: Bare Demo of IEEEtran.cls for IEEE Journals}
% The only time the second header will appear is for the odd numbered pages
% after the title page when using the twoside option.
% 
% *** Note that you probably will NOT want to include the author's ***
% *** name in the headers of peer review papers.                   ***
% You can use \ifCLASSOPTIONpeerreview for conditional compilation here if
% you desire.




% If you want to put a publisher's ID mark on the page you can do it like
% this:
%\IEEEpubid{0000--0000/00\$00.00~\copyright~2015 IEEE}
% Remember, if you use this you must call \IEEEpubidadjcol in the second
% column for its text to clear the IEEEpubid mark.



% use for special paper notices
%\IEEEspecialpapernotice{(Invited Paper)}




% make the title area
\maketitle

% As a general rule, do not put math, special symbols or citations
% in the abstract or keywords.
\begin{abstract}
Abstract goes here.
\end{abstract}

% Note that keywords are not normally used for peerreview papers.
\begin{IEEEkeywords}
neutron spectroscopy, fission chambers, maximum entropy
\end{IEEEkeywords}






% For peer review papers, you can put extra information on the cover
% page as needed:
% \ifCLASSOPTIONpeerreview
% \begin{center} \bfseries EDICS Category: 3-BBND \end{center}
% \fi
%
% For peerreview papers, this IEEEtran command inserts a page break and
% creates the second title. It will be ignored for other modes.
\IEEEpeerreviewmaketitle


\section{Introduction}

\IEEEPARstart{O}f interest to any reactor physicist is a high resolution, spatial, temporal, and spectral representation of the in-core neutron flux of any reactor.
Within recent years, scientists at Kansas State University have made many developments in fission chamber technology, achieving goals like the reduction of local flux perturbation, point-like detector geometry and extreme widening of the power band \cite{reichenberger2016micro} \cite{reichenberger2018fabrication}.
Currently, these technologies, known as Micro-Pocket Fission Detectors, only employ a single fission chamber at each location which allow for detection of the total flux.
However, one can imagine the co-location of multiple chambers in the same nodal region, each containing a unique fissionable isotope, providing many more, and ideally independent, responses to the in-core spectral flux.
Although useful, there exist limitations on the ammount of information that can be extracted from a neutron signal using an increased number of fission chambers.
Geometric constraints and isotopic availability and manufacturing exist as two major limitations.
Another issue, the problem considered in this particular work, deals with the transformation of response data into the spectral neutron flux, specifically, the underdetermined problem of spectral deconvolution caused by both the uncertainty in the responses and the disparity between the response vector and the desired, larger neutron spectral flux vector.


\section{Spectral Deconvolution}

Spectral deconvolution, also known as spectral unfolding, is a problem commonly found within neutron spectrometry.
It arises in part by the desire to produces a high resolution representation of the neutron flux spectrum featuring useful detail in special regions.
The equation governing the process, when represented as a summation, is expressed as follows:

\begin{equation}
  r_i + \epsilon_i = \sum_{g=1}^{N} \Sigma_{fg,i}  \phi_g\, , 
    \qquad i = 1 \ldots N \, ,
\label{eq:response}
\end{equation}

where each $r_i$ is the response of isotope $i$, $\epsilon_i$ is the error associated with $r_i$, $\Sigma_{fg,i}$ is the flux-weighted, integrate fission cross section for energy group $g$ for isotope $i$, and $\phi_g$ is the integrated flux in energy group $g$.
Known in this equation are the responses $r_i$, obtained from detector measurements, and assumed is the matrix $\Sigma$.
It is assumed because to flux-weight the cross section requires {\it a priori} knowledge of the true flux spectrum, the obtention of which is the goal of our work here.
Deconvolution refers to the class of methods used to obtain the flux vector.

Many different methods of deconvolution have been studied and shown to have varying levels of largely application-dependent success.
Two popular methods used in similar applications to this are included in the RSICC Peripheral Science Routine Collection, namely, MAXED and Gravel.

MAXED, arguably the more sophisticated of the two finds its basis in information theory and leverages the principle of maximum entropy \cite{reginatto1999maxed}.
It seeks to solve a system of equations comprised of \EQ{eq:response} and \EQ{eq:omega} while maximizing \EQ{eq:entropy}

\begin{equation}
  \sum \frac{\epsilon_i^2}{\sigma_k^2} = \Omega
\label{eq:omega}
\end{equation}

\begin{equation}
  S = - \sum [f_i \ln(f_i/f_i^{DEF}) + f_i^{DEF} - f_i]
\label{eq:entropy}
\end{equation}

This maximization of $S$ is done through use of simulated annealing and detailed in Reginatto and Goldhagen \cite{reginatto1999maxed}.
The closed form solution allows for the propogation of uncertainty, a feature not available in other, iterative-type deconvolution methods.

Gravel employs a modified version of the Sand II algorithm which was first detailed in a 1967 report from the Air Force Weapons Laboratory \cite{mcelroy1967computer}.
The following equation describes the (relatively simple) algorithm:

\begin{equation}
 f_i^{J+1} = f_i^{J} \exp(\frac{\sum_k W_{ik}^J \log(\frac{N_k}{\sum_{i'} R_{ki'}f_{i'}^J})}{\sum_k W_{ik}^J}) 
\label{eq:sand}
\end{equation}

where $N_k$ is the measured response, $\sigma_k$ is the measurement error, $R_{ki}$ is the response matrix, and $W_{ik}^J$ is a weighting term:

\begin{equation}
 W_{ik}^J = \frac{R_{ki} f_i^J}{\sum_{i'} f_{i'}^J} \frac{N_k^2}{\sigma_k^2} 
\label{eq:weighting}
\end{equation}

$J$ represents the current iteration, $i$ is the energy group index, and $k$ is the response index.
Because there is no closed form solution to the algorithm, uncertainty propagation is not possible here \cite{reginatto2004umg}.



\section{Methods}

\begin{figure}[h!tb]
  \centering
  \includegraphics[width = 0.49\textwidth]{../plot/umg_ds_gr}
  \caption{The default spectra used in unfolding.}
  \label{fig:default_spectra}
\end{figure}

A range of fissionable isotopes are theoretically available for use in fission chambers and so various isotopic groupings, found in a \TAB{tab:isos}, unfolding was performed.
Because both of these algorithms require the use of a default spectrum, the interest in an analysis of this problem comes from the ability of the algorithms to recieve some typical flux spectrum and accurately reproduce a true flux spectrum.
An MCNP simulation was used to produce a representative flux spectrum for the KSU Triga Mark II research reactor and then detector responses were obtained by folding (integrating) this flux with the isotopic cross sections as found in the above equation.
Then, using four unique default spectra shown in \FIG{fig:default_spectra}, both unfolding methods were applied to each isotopic set and solution flux spectra were obtained.
Three of the default spectra were textbook PWR flux spectra with different magnitudes.
One was scaled to identical integrated flux as the true, TRIGA spectrum, the other two were multiplied by a scaling coefficient of 0.5 or 2.0, representing the theoretical range of mis-guessing the TRIGA's true absolute flux.
The other, uniquely-shaped spectrum was obtained by setting each bin's integrated flux to unity then scaling the overall flux to match that of the true spectrum.
The Wims69 group structure was used for this analysis.

\begin{table*}[h]
\centering
\begin{tabular}{ |c|c| } \label{tab:isos}
 \hline
 \textbf{Case} & \textbf{Included Isotopes} \\
 \hline
 Case 1 & ${}^{235}$U, ${}^{238}$U, and ${}^{232}$Th \\
 \hline
 Case 2 & Case 1 + ${}^{237}$Np and ${}^{238}$Pu \\
 \hline
 Case 3 & Case 2 + ${}^{239}$Pu, ${}^{240}$Pu, and ${}^{241}$Pu \\
 \hline
 Case 4 & Case 3 + ${}^{233}$U, ${}^{234}$U, ${}^{238}$Pu, ${}^{242}$Pu, ${}^{241}$Am, ${}^{242m}$Am, ${}^{244}$Cm, and ${}^{245}$Cm \\
 \hline
 Case 5 & Case 3 + ${}^{233}$U${}^{113}$Cd, ${}^{235}$U${}^{113}$Cd, ${}^{238}$Pu${}^{113}$Cd, ${}^{239}$Pu${}^{113}$Cd, ${}^{241}$Pu${}^{113}$Cd \\
 \hline
 Case 6 & Case 3 + ${}^{233}$U${}^{155}$Gd, ${}^{235}$U${}^{155}$Gd, ${}^{238}$Pu${}^{155}$Gd, ${}^{239}$U${}^{155}$Gd, ${}^{241}$Pu${}^{155}$Gd \\
 \hline
 Case 7 & Case 3 + all Cd and Gd shielded isotopes. \\
 \hline
\end{tabular}
\end{table*}


\section{Results and Discussion}

\subsection{Default Spectra}

The unity spectrum stands out as the worst of the default spectra, exhibiting very noisy behavior between groups 20 and 40 and thus make it least useful for use to find a solution spectrum.
This is due in part to the group structure used.
The variance in bin width across the Wims 69 group structure forms peak behavior in this region when producing a unity spectrum that translates to inaccuracies in the unfolding process.
It is conceivable, then, that a structure with more uniformity across the spectrum might correlate to a smoother solution spectrum.

A second, obvious feature within the Gravel results in \FIG{fig:ds_sol_gr} is the indifference to starting total flux, given an identical flux spectrum.
This contrasts with MAXED's clear dependence on integral flux, exibiting it's lowest average error for the low flux spectrum.
Another feature of interest is MAXED's adherence to the default spectra through the region bounded by approximately $10^{-3}$ and $5 \times 10^{-1}$.
The maximum entropy principle discourages deviation from the default spectrum, and therefore is likely to exibit a consistent and unfavorable bias throughout this region for a wide range of default spectra.

Although trouble for the unity spectrum, all default spectra struggle to match the true spectrum within groups 20 through 40.
The low flux spectrum combined with MAXED outperforms any other combination within this section of analysis, highlighting the importance of default spectrum selection.


\begin{figure}[h!tb]
  \centering
  \includegraphics[width = 0.49\textwidth]{../plot/umg_ds_sol_gr}
  \caption{The solutions provided by Gravel given the varied default spectra.}
  \label{fig:ds_sol_gr}
\end{figure}

\begin{figure}[h!tb]
  \centering
  \includegraphics[width = 0.49\textwidth]{../plot/umg_ds_sol_err_gr}
  \caption{The group-wise error produced by Gravel given the varied default spectra.}
  \label{fig:ds_err_gr}
\end{figure}

\begin{figure}[h!tb]
  \centering
  \includegraphics[width = 0.49\textwidth]{../plot/umg_ds_sol_mx}
  \caption{The solutions provided by MAXED given the varied default spectra.}
  \label{fig:ds_sol_mx}
\end{figure}

\begin{figure}[h!tb]
  \centering
  \includegraphics[width = 0.49\textwidth]{../plot/umg_ds_sol_err_mx}
  \caption{The group-wise error produced by MAXED given the varied default spectra.}
  \label{fig:ds_err_mx}
\end{figure}


\subsection{Isotopic Groups}

For the subsequent analysis, the `same flux' spectrum was used as the default spectrum.
For both programs, the addition of more responses generally leads to a decrease in the group-wise mean error.
However, when the neptunium and thorium resonses are included in the analysis, case 2 actually leads to an increase in error, visible in \FIG{fig:isos_err_gr} and \FIG{isos_err_mx}.
This is likely due to the additional element's energy region of highest response.
They add a a degree of importance to capturing the detail in the lower end of the 1/E region and as a result, cause the erratic behavior while attempting to fix the solution spectrum.
MAXED generally outperforms gravel for the few response cases.
The matching along the entire 1/E region is better for MAXED, causing this outperformance.
It should be noted, however, that Gravel exhibits a higher capacity to capture detail, at lease in these cases: error is virtually non-existent in the high-energy, fission spectrum end of the curve, and the curvature around the low-energy, maxwellian region is more extreme, to match the true spectrum.
Restated, MAXED operates on a maximum entropy principle, so the spectrum is often less free than Gravel's to undergo more extreme shape changes in attempting to locate a solution spectrum.



\begin{figure}[h!tb]
  \centering
  \includegraphics[width = 0.49\textwidth]{../plot/umg_isos_gr}
  \caption{The solutions provided by Gravel given the varied isotopic cases.}
  \label{fig:isos_gr}
\end{figure}

\begin{figure}[h!tb]
  \centering
  \includegraphics[width = 0.49\textwidth]{../plot/umg_isos_err_gr}
  \caption{The group-wise error produced by Gravel given the varied isotopic cases.}
  \label{fig:isos_err_gr}
\end{figure}

\begin{figure}[h!tb]
  \centering
  \includegraphics[width = 0.49\textwidth]{../plot/umg_isos_mx}
  \caption{The solutions provided by MAXED given the varied  isotopic cases.}
  \label{fig:isos_mx}
\end{figure}

\begin{figure}[h!tb]
  \centering
  \includegraphics[width = 0.49\textwidth]{../plot/umg_isos_err_mx}
  \caption{The group-wise error produced by MAXED given the varied  isotopic cases.}
  \label{fig:isos_err_mx}
\end{figure}

\subsection{Shielded Responses}

It is with the inclusion of shielded responses that Gravel really begins to exhibit its usefulness.
Both programs continue to find improved spectra given additional repsponses, culminating in a value of 18.5\% by Gravel including the Cd and Gd responses.
For both programs, the Gd inclusion outperforms the Cd included responses, a pattern which does not hold true for other unfolding techniques \cite{roberts2018use}.

\begin{figure}[h!tb]
  \centering
  \includegraphics[width = 0.49\textwidth]{../plot/umg_shielded_gr}
  \caption{The solutions provided by Gravel given the varied isotopic cases, including the shielded responses.}
  \label{fig:shielded_gr}
\end{figure}

\begin{figure}[h!tb]
  \centering
  \includegraphics[width = 0.49\textwidth]{../plot/umg_shielded_err_gr}
  \caption{The group-wise error produced by Gravel given the varied isotopic cases, including the shielded responses.}
  \label{fig:shielded_err_gr}
\end{figure}

\begin{figure}[h!tb]
  \centering
  \includegraphics[width = 0.49\textwidth]{../plot/umg_shielded_mx}
  \caption{The solutions provided by MAXED given the varied isotopic cases, including the shielded responses.}
  \label{fig:shielded_mx}
\end{figure}

\begin{figure}[h!tb]
  \centering
  \includegraphics[width = 0.49\textwidth]{../plot/umg_shielded_err_mx}
  \caption{The group-wise error produced by MAXED given the varied isotopic cases, including the shielded responses.}
  \label{fig:shielded_err_mx}
\end{figure}



\section{Conclusion}

These are the best we got.

 
% use section* for acknowledgment
%\section*{Acknowledgment}

%Shoutout to my bros.

% Can use something like this to put references on a page
% by themselves when using endfloat and the captionsoff option.
\ifCLASSOPTIONcaptionsoff
  \newpage
\fi



% trigger a \newpage just before the given reference
% number - used to balance the columns on the last page
% adjust value as needed - may need to be readjusted if
% the document is modified later
%\IEEEtriggeratref{8}
% The "triggered" command can be changed if desired:
%\IEEEtriggercmd{\enlargethispage{-5in}}

% references section

% can use a bibliography generated by BibTeX as a .bbl file
% BibTeX documentation can be easily obtained at:
% http://mirror.ctan.org/biblio/bibtex/contrib/doc/
% The IEEEtran BibTeX style support page is at:
% http://www.michaelshell.org/tex/ieeetran/bibtex/
\bibliographystyle{IEEEtran}
% argument is your BibTeX string definitions and bibliography database(s)
%\bibliography{IEEEabrv,../bib/paper}
%
% <OR> manually copy in the resultant .bbl file
% set second argument of \begin to the number of references
% (used to reserve space for the reference number labels box)
% \begin{thebibliography}{1}

% \bibitem{IEEEhowto:kopka}
% H.~Kopka and P.~W. Daly, \emph{A Guide to \LaTeX}, 3rd~ed.\hskip 1em plus
%   0.5em minus 0.4em\relax Harlow, England: Addison-Wesley, 1999.

% \end{thebibliography}
\bibliography{refs}
% biography section
% 
% If you have an EPS/PDF photo (graphicx package needed) extra braces are
% needed around the contents of the optional argument to biography to prevent
% the LaTeX parser from getting confused when it sees the complicated
% \includegraphics command within an optional argument. (You could create
% your own custom macro containing the \includegraphics command to make things
% simpler here.)
%\begin{IEEEbiography}[{\includegraphics[width=1in,height=1.25in,clip,keepaspectratio]{mshell}}]{Michael Shell}
% or if you just want to reserve a space for a photo:





% insert where needed to balance the two columns on the last page with
% biographies
%\newpage

% You can push biographies down or up by placing
% a \vfill before or after them. The appropriate
% use of \vfill depends on what kind of text is
% on the last page and whether or not the columns
% are being equalized.

%\vfill

% Can be used to pull up biographies so that the bottom of the last one
% is flush with the other column.
%\enlargethispage{-5in}



% that's all folks
\end{document}


